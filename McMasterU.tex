\documentclass[12pt]{beamer}

% Specify theme
\usetheme{McMasterU}

% Start Changes by Ricardo
\setbeamercolor{section in head/foot}{fg=white,bg=black}

\makeatletter
\setbeamertemplate{headline}{%
    \begin{beamercolorbox}[ht=2.25ex,dp=3.75ex]{section in head/foot}
        \insertnavigation{\paperwidth}
    \end{beamercolorbox}%
}%
\makeatother
% End Changes by Ricardo

 %\setbeamertemplate{footline}[frame number]{} % Uncomment this line if you want to remove the footer from each slide (and replace it with just the slide number (X/Y) in the bottom right of each slide.

%===============================================================%
% 				BEGIN YOUR PRESENTATION HERE					%
%===============================================================%

% Title and author information
\title[Short title]{Your Presentation Title}
\author{Your Name}
\institute[]{McMaster University}
\date{\today}


%  \usepackage[sfmath]{kpfonts}
%  \renewcommand*\familydefault{\sfdefault}

%\setbeamerfont{frametitle}{shape=\scshape}

%===============================================================%
\begin{document}
%===============================================================%

\maketitle



%===============================================================%
\section{Lists}
%===============================================================%
\subsection{}
\begin{frame}{Itemized List}

	Itemized lists are punctuated by little shields

	\begin{itemize}
		\item Item
		\item Item
			\begin{itemize}
				\item Sub-item
				\item Sub-item
			\end{itemize}
		\item Item
	\end{itemize}

\end{frame}



\begin{frame}{Enumerate}

	\begin{enumerate}
		\item Item
		\item Item
			\begin{enumerate}
				\item Sub-item
			\end{enumerate}
	\end{enumerate}

\end{frame}



%===============================================================%
\section{Animations}
%===============================================================%
\subsection{}
\begin{frame}{Slide animation}

	Sometimes you want to hide later text/elements of a particular slide to keep the focus on the early part of the slide.

	\bigskip

	\onslide<2>{By having the text shaded out (and not completely missing), your audience can see that you do have some more information that will come shortly.}

\end{frame}



%===============================================================%
\section{Blocks}
%===============================================================%
\subsection{}
\begin{frame}{Blocks}

	\begin{block}{Regular Block}
		Text goes here
	\end{block}

	\begin{alertblock}{Alert Block}
		Stands out a bit more
	\end{alertblock}

	\begin{exampleblock}{Example Block}
		Also stands out $y=\beta x+ \varepsilon$
	\end{exampleblock}

\end{frame}



%===============================================================%
\appendix
%===============================================================%

{\BackgroundShaded
\begin{frame}
% BLANK FRAME AT THE END
\end{frame}
}



%===============================================================%
\section{First appendix section}
%===============================================================%

\begin{frame}{Appendix sample}

	Note that this slide doesnt count towards the total slides shown in the regular presentation

\end{frame}



%===============================================================%
\end{document}
%===============================================================% 